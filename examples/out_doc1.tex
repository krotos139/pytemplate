\documentclass[12pt,a4paper]{scrartcl}
\usepackage[utf8x]{inputenc}
\usepackage[english,russian]{babel}
\usepackage{indentfirst}
\usepackage{misccorr}
\usepackage{graphicx}
\usepackage{amsmath}
\usepackage{array}
\usepackage{longtable}
\usepackage{graphics}
\usepackage{xspace}
\usepackage{fancyhdr}

\graphicspath{{img/}}

\def\@xobeysp{ }

\pagestyle{fancy}
\renewcommand{\headrulewidth}{0.4pt}
\renewcommand{\footrulewidth}{0.4pt}
\fancyhead[L]{Некая фирма}
\fancyhead[R]{Пример}
\fancyfoot[L]{Документ1}
\fancyfoot[R]{\thepage}
\fancyfoot[C]{}
\fancyheadoffset{0mm}
\fancyfootoffset{0mm}
\setlength{\headheight}{17pt}


\begin{document}

\section{Документ}
Пример тенерации tex документа.

\section{Контент из XML}
\input{out_doc1_part1}

\section{Контент из CSV}
\input{out_doc1_part2}

\section{Контент из SQLite}
\input{out_doc1_part3}

\section{Контент из текстового файла}
\input{out_doc1_part4}

\section{Контент из Файловой системы}
\input{out_doc1_part5}

\end{document}
